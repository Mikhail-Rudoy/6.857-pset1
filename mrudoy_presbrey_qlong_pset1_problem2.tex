\documentclass[11pt]{article}

\newcommand{\team}{Qian Long \\ Joseph W Presbrey \\ Mikhail Rudoy }
\newcommand{\ps}{ Problem Set 1 }

%\pagestyle{headings}
\usepackage[dvips]{graphics,color}
\usepackage{amsfonts}
\usepackage{amssymb}
\usepackage{amsmath}
\usepackage{latexsym}
\setlength{\parskip}{1pc}
\setlength{\parindent}{0pt}
\setlength{\topmargin}{-3pc}
\setlength{\textheight}{9.5in}
\setlength{\oddsidemargin}{0pc}
\setlength{\evensidemargin}{0pc}
\setlength{\textwidth}{6.5in}

\newcommand{\answer}[1]{
\newpage
\noindent
\framebox{
	\vbox{
		6.857 Homework \hfill {\bf \ps} \hfill \# #1  \\
		\team \hfill \today
	}
}
\bigskip

}


\begin{document}
\answer{1-2. One-time pad}

\section*{1-2a)}
Given two ciphertext $C_1$ and $C_2$ encoding using the same one-time-pad $P$, we can deduce the original messages $M_1$ and $M_2$ by xoring the ciphertexts, as follows.

\begin{eqnarray*}
M_1 \oplus P &=& C_1\\
M_2 \oplus P &=& C_2\\
M_1 \oplus M_2 &=& C_1 \oplus C_2\\
\end{eqnarray*}

To find the secret words we can first compute $C_1 \oplus C_2$ using the given values. Then, using the above relation, we simply have to find a pair of 8-character words $M_1$ and $M_2$ such that $M_1 \oplus M_2$ equals the computed value.

Given the assumption that the secret words are common 8-character English words, we take a list of common 8-character words from the internet and wrote a python script to look for the pair of words that, when xor-ed together, equals $C_1 \oplus C_2$.
We optimize the performance of the search by first computing $M_1 \oplus C_1 \oplus C_2 = M_2$ for each $M_1$ in the list of words and storing the results in a hash table. Then we scan the word list again and simply check if the each word is in the hash table, taking $O(1)$ time for each check. If we find a word in the hashtable, then the pair $M_1$ and $M_2$ are found. Thus, this script runs in linear time.

The two words are security and networks.
\newpage
Python code:
\begin{verbatim}
def ascii_word(string):
  output = []
  for char in string:
    output.append(ord(char))
  return output


def xor_word(array1, array2):
  output = []
  for i in xrange(len(array1)):
    output.append(array1[i] ^ array2[i])
  return output

def main():
  c1 = [0xe9, 0x3a, 0xe9, 0xc5, 0xfc, 0x73, 0x55, 0xd5]
  c2 = [0xf4, 0x3a, 0xfe, 0xc7, 0xe1, 0x68, 0x4a, 0xdf]
  result = xor_word(c1, c2)
  # str rep of ascii bytes
  resultstr = ''.join([str(x) for x in result])

  f = open('words.txt', 'r')
  words = []
  for line in f.readlines():
    words.extend(line.strip().split(' '))

  # find 2 words such that w1 ^ w2 == result
  # put result ^ w into hashtable => result ^ w1 = w2
  hashtable = {}
  for i in xrange(len(words)):
    m2 = ''.join([str(x) for x in xor_word(result, ascii_word(words[i]))])
    hashtable[m2] = words[i]

  for i in xrange(len(words)):
    ascii_str = ''.join([str(x) for x in ascii_word(words[i])])
    if ascii_str in hashtable:
      print "found match:"
      print hashtable[ascii_str]
      print words[i]
      break
  return

if __name__ == "__main__":
  main()
\end{verbatim}

\newpage
\section*{1-2b)}
\end{document}


