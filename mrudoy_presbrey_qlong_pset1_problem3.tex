\documentclass[11pt]{article}

\newcommand{\team}{Qian Long \\ Joseph W Presbrey \\ Mikhail Rudoy }
\newcommand{\ps}{ Problem Set 1 }

%\pagestyle{headings}
\usepackage[dvips]{graphics,color}
\usepackage{amsfonts}
\usepackage{amsthm}
\usepackage{amssymb}
\usepackage{amsmath}
\usepackage{latexsym}
\setlength{\parskip}{1pc}
\setlength{\parindent}{0pt}
\setlength{\topmargin}{-3pc}
\setlength{\textheight}{9.5in}
\setlength{\oddsidemargin}{0pc}
\setlength{\evensidemargin}{0pc}
\setlength{\textwidth}{6.5in}

\newtheorem{Lemma}{Lemma}

\newcommand{\answer}[1]{
\newpage
\noindent
\framebox{
	\vbox{
		6.857 Homework \hfill {\bf \ps} \hfill \# #1  \\ 
		\team \hfill \today
	}
}
\bigskip

}


\begin{document}
\answer{1-3. Detecting Pad Reuse}

\section*{1-3a)}

Define $R(a,b)$ to be the length of the longest bitstring that is a substring of both the bittexts $a$ and $b$. Define $S(a,b)$ to be the length of the longest bitstring that is a substring of both the bitstrings $a$ and $b$ in the same positions. 

First we will prove two lemmas: 

\begin{Lemma}
Suppose we are given two bitstrings $a$ and $b$ of length $l\le k$ that satisfy $\mathbb{P}(a = s) = \frac{1}{2^l}$ and $\mathbb{P}(b = s) = \frac{1}{2^l}$ for every $s \in \{0, 1\}^l$. Then for any asymptotically positive polynomial $q(k)$, we have $\mathbb{P}\bigg( S(a,b) > \log_2(k)+\log_2\ln(k)\bigg) < \frac{1}{q(k)}$ provided $k$ is sufficiently large.
\end{Lemma}\begin{proof}
The condition on $a$ and $b$ implies that the xor of the two ciphertexts, $a \oplus b$, can be any bitstring of length $l$ with uniform probability. A bit of $a \oplus b$ is $0$ if and only if the corresponding bits of $a$ and $b$ are the same. The value $S(a,b)$ is the length of the longest sequence of bits in $a$ such that the corresponding sequence of bits in $b$ is the same. In other words, $S(a,b)$ is the length of the longest run of $0$s in $a \oplus b$. 

Since $a \oplus b$ is chosen uniformly at random, we can pretend that it was generated by a sequence of $l$ coin flips. We identify $0$s in $a \oplus b$ with coins landing heads.

Consider first the case that $l = k$. In this case, the useful fact given in the problem applies. The given fact asserts (in this case) that for any asymptotically positive polynomial $q(k)$,  
$$\mathbb{P}\bigg(\log_2(k) - \log_2\ln\ln(k) \le S(a,b) \le log_2(k) + \log_2\ln(k)\bigg) \ge 1-\frac{1}{q(k)}$$
 provided $k \ge K$ (for some $K$ dependent only on $q$). Then 
$$\mathbb{P}\bigg(\log_2(k) - \log_2\ln\ln(k) > S(a,b) \text{ or } S(a,b) > log_2(k) + \log_2\ln(k)\bigg) < \frac{1}{q(k)}$$
and finally 
$$\mathbb{P}\bigg(S(a,b) > log_2(k) + \log_2\ln(k)\bigg) < \frac{1}{q(k)}$$
Thus in the case $l = k$, our desired result is proved. 

For every case where $l \ne k$, the number of coin flips is smaller, and so the probability of a long run of heads can only decrease from the $l = k$ case. Thus we have shown that $$\mathbb{P}\bigg(S(a,b) > log_2(k) + \log_2\ln(k)\bigg) < \frac{1}{q(k)}$$
for all cases, provided $k$ is sufficiently large. \end{proof}

\begin{Lemma}
Given that two ciphertexts $u$ and $v$ of length $n$ are properly encrypted, for any asymptotically positive polynomial $q(n)$, we have $\mathbb{P}\bigg( R(u,v) > \log_2(n)+\log_2\ln(n)\bigg) < \frac{1}{q(n)}$ provided $n$ is sufficiently large.
\end{Lemma}\begin{proof}
Let $u_{[-i]}$ and $v_{[-i]}$ denote, for non-negative $i$, the substring of $u$ or $v$ which excluding the first $i$ bits. Similarly, define $u_{[i]}$ and $v_{[i]}$ to be, for non-negative $i$, the substring which excludes the last $i$ bits. 

Let $s$ be the longest substring of both $u$ and $v$. $R(u,v)$ is then the length of $s$. Suppose $s$ begins at index $j_u$ in $u$ and at index $j_v$ in $v$, let $i = j_v - j_u$. Consider $u_{[i]}$ and $v_{[-i]}$. In all cases, the first $|i|$ bits are removed from the string in which the index where $s$ begins is larger. The last $|i|$ bits are removed from the other string. The result is that both new strings contain $s$ at the same range of indices, and both strings have the same length. Thus $S(u_{[i]},v_{[-i]}) \ge R(u,v)$ for some choice of $i$ where $-n \le i \le n$. We can conclude that $R(u,v) \le \max\limits_{-n \le i \le n}\bigg(S(u_{[i]},v_{[-i]})\bigg)$. 

Since $u$ and $v$ are properly encrypted, we know that $\mathbb{P}(u_{[i]} = s) = \mathbb{P}(v_{[-i]} = s) = \frac{1}{2^{n-|i|}}$ for every $s \in \{0, 1\}^{n-|i|}$ and for every $i$ with $-n \le i \le n$. In addition, $u_{[i]}$ and $v_{[-i]}$ always have length at most $n$, so we can apply Lemma 1. The lemma tells us that for any asymptotically positive polynomial $p$, provided $n$ is sufficiently large, $\mathbb{P}\bigg( S(u_{[i]},v_{[-i]}) > \log_2(n)+\log_2\ln(n)\bigg) < \frac{1}{p(n)}$. Taking $p(n) = (2n+1)q(n)$, we see that provided $n$ is sufficiently large, $\mathbb{P}\bigg( S(u_{[i]},v_{[-i]}) > \log_2(n)+\log_2\ln(n)\bigg) < \frac{1}{(2n+1)q(n)}$.

Since $R(u,v) \le \max\limits_{-n \le i \le n}\bigg(S(u_{[i]},v_{[-i]})\bigg)$, we know that 
$$\mathbb{P}\bigg(R(u,v)> \log_2(n)+\log_2\ln(n)\bigg) \le \mathbb{P}\bigg( \max\limits_{-n \le i \le n}\bigg(S(u_{[i]},v_{[-i]})\bigg) > \log_2(n)+\log_2\ln(n)\bigg)$$
$$=\mathbb{P}\left(\begin{matrix} S(u_{[-n]},v_{[n]}) > \log_2(n)+\log_2\ln(n) & \text{or} \\ S(u_{[-(n - 1)]},v_{[(n-1)]}) > \log_2(n)+\log_2\ln(n) & \text{or} \\ ... & \text{or}\\ S(u_{[(n - 1)]},v_{[-(n-1)]}) > \log_2(n)+\log_2\ln(n) & \text{or} \\ S(u_{[n]},v_{[-n]}) > \log_2(n)+\log_2\ln(n)  \end{matrix}\right)$$
$$\le \mathbb{P}\bigg( S(u_{[n]},v_{[-n]}) > \log_2(n)+\log_2\ln(n)\bigg) + ... + \mathbb{P}\bigg( S(u_{[-n]},v_{[n]}) > \log_2(n)+\log_2\ln(n)\bigg)$$
$$\le \frac{1}{(2n+1)q(n)} + \frac{1}{(2n+1)q(n)} + ... + \frac{1}{(2n+1)q(n)} = \frac{(2n+1)}{(2n+1)q(n)} = \frac{1}{q(n)}$$

We have shown that $\mathbb{P}\bigg( R(u,v) > \log_2(n)+\log_2\ln(n)\bigg) < \frac{1}{q(n)}$ provided $n$ is sufficiently large, exactly as desired.\end{proof}

Let $l(n)$ be the number of ciphertexts we are given. Let $r(n)$ be any asymptotically positive polynomial. 

What we are interested is the value
$$\mathbb{P}\left(\max\limits_{1\le i < j \le l(n)}R(c_i,c_j) \le \log_2(n)+\log_2\ln(n)\right)$$
and in particular, we wish to show that it is greater than $1-\frac{1}{r(n)}$ for $n\ge n_0$ for some $n_0$.

Alternatively, since $$\mathbb{P}\left(\max\limits_{1\le i < j \le l(n)}R(c_i,c_j) > \log_2(n)+\log_2\ln(n)\right) = 1-\mathbb{P}\left(\max\limits_{1\le i < j \le l(n)}R(c_i,c_j) \le \log_2(n)+\log_2\ln(n)\right)$$
we must simply show that $\mathbb{P}\left(\max\limits_{1\le i < j \le l(n)}R(c_i,c_j) > \log_2(n)+\log_2\ln(n)\right) < \frac{1}{r(n)}$ holds for $n\ge n_0$ for some $n_0$.

In addition, we know that $$\mathbb{P}\left(\max\limits_{1\le i < j \le l(n)}R(c_i,c_j) > \log_2(n)+\log_2\ln(n)\right) =  \mathbb{P}\left(\begin{matrix} R(c_1,c_2) > \log_2(n)+\log_2\ln(n) & \text{or} \\ R(c_1,c_3) > \log_2(n)+\log_2\ln(n) & \text{or} \\ R(c_2,c_3) > \log_2(n)+\log_2\ln(n) & \text{or} \\ ... & \text{or} \\ R(c_{l(n)-1},c_{l(n)}) > \log_2(n)+\log_2\ln(n)  \end{matrix}\right)$$ $$\le \sum_{i=1}^{l(n)-1}\sum_{j=i+1}^{l(n)}\mathbb{P}\bigg( R(c_i,c_j) > \log_2(n)+\log_2\ln(n)\bigg)$$

By Lemma 2, $\mathbb{P}\bigg( R(c_i,c_j) > \log_2(n)+\log_2\ln(n)\bigg) < \frac{1}{q(n)}$ for every $i$, $j$, and asymptotically positive polynomial $q(n)$, provided $n\ge n_0$ for some $n_0$ dependent only on $q$. Let $q(n) = \frac{1}{2}r(n)l(n)(l(n)-1)$, and let $N$ be the associated value of $n_0$. Then for $n \ge N$ we see that $$\sum_{i=1}^{l(n)-1}\sum_{j=i+1}^{l(n)}\mathbb{P}\bigg( R(c_i,c_j) > \log_2(n)+\log_2\ln(n)\bigg) <\sum_{i=1}^{l(n)-1}\sum_{j=i+1}^{l(n)}\frac{1}{q(n)}$$
$$= \frac{l(n)(l(n) - 1)}{2}\times\frac{1}{q(n)}=\frac{l(n)(l(n) - 1)}{2q(n)}=\frac{l(n)(l(n) - 1)}{2\frac{1}{2}r(n)l(n)(l(n)-1)} = \frac{1}{r(n)}$$

We can conclude that $\mathbb{P}\left(\max\limits_{1\le i < j \le l(n)}R(c_i,c_j) > \log_2(n)+\log_2\ln(n)\right) < \frac{1}{r(n)}$ for $n\ge N$, so $$\mathbb{P}\left(\max\limits_{1\le i < j \le l(n)}R(c_i,c_j) \le \log_2(n)+\log_2\ln(n)\right) > 1- \frac{1}{r(n)}$$ for $n\ge N$. Since $r(n)$ can be any asymptotically positive polynomial, we see that the length of the longest repeated bitstring $\bigg(\max\limits_{1\le i < j \le l(n)}R(c_i,c_j)\bigg)$ is, with high probability, at most $\log_2(n)+\log_2\ln(n)$, as desired.

\newpage

\section*{1-3b)}

Since the bound from part a only applies in the asymptotic limit, it is not significant to evaluate it at small inputs and compare it to some other function. We wish instead to compare its asymptotic behavior to that of the average length of the longest run of identical bits in pairs of English passages.

Consider the given graph. Let the function being represented be expressed as $y = f(x)$. Here $x$ = $\ln n$, and $y$ is the average length of the longest run of characters in the same positions of two English passages. We wish to discuss $\hat{y} = \hat{f}(x)$, the average length of the longest run of identical bits. Let $l$ be the length of the longest run of identical characters in some particular two passages, and let $\hat{l}$ be the length of the longest run of identical bits in the same two passages. We note that each character is composed of $8$ bits, so, as a result, it is always true that $8l \le \hat{l} \le 8l + 14$. Since $y$ is the average value of $l$ over some sample and $\hat{y}$ is the average value of $\hat{l}$ over the same sample, we see that $8y \le \hat{y} \le 8y + 14$. We can simply say that $\hat{f}(x) = \hat{y} \approx 8y = 8f(x)$, and that $\Theta(\hat{f}(x)) = \Theta(f(x))$. From the graph we see that $f$ is super-linear, so $f(x) = \omega(x)$, and as a result, so is $\hat{f(x)}$.

Let $g(x)$ be the bound found in part a, expressed in terms of $x$. $g(x) = \log_2(n)+\log_2\ln(n) = \log_2(e)\times\ln(n)+\log_2\ln(n) = x\times\log_2(e) + \log_2(x)$. We know that $x\times\log_2(e) = \Theta(x)$ and $\log_2(x) = o(x)$, so $g(x) = \Theta(x) + o(x) = \Theta(x)$.

We have seen that $g(x) = \Theta(x)$, and that $\hat{f}(x) = \omega(x)$, so $\hat{f}(x) = \omega(g(x))$. In other words, for sufficiently large $x$, the value of $\hat{f}(x)$ will exceed the value of $g(x)$, and pairs of English texts will, on average, have longer common sub-bitstrings in corresponding positions than the longest common substrings (in any position) of pairs of correctly encrypted crptotexts. 

\newpage

\section*{1-3c)}

\begin{Lemma}
A suffix tree can be generated from a string in linear time.
\end{Lemma}
\begin{proof}
TODO
\end{proof}

We are given $l(n) = poly(n)$ $n$-bit cyphertexts ($c_0$ through $c_{l(n) -1}$) with one instance of pad reuse. Let $\Sigma$ be the following alphabet: $\{0,1\}\cup \{\$0, \$1, \$2, ..., \$(l(n) - 1)\}$. In $\Sigma$, $\$i$ is treated as a single character. We will construct a string $s$ of length $l(n)\times (n+1)$ by the following rule: $s = s_0\|s_1\|...\|s_{l(n) -1}$, where $s_i = c_i \| \$i$.

The next step is to generate a suffix tree. By Lemma 3, this can be done in linear time. 

We can also find the deepest non-leaf node in the tree in linear time. The sequence of nodes from the root to this node correspond with the longest substring repeated in $s$. Call this substring $t$. The characters $\$i$ appear exactly once in $s$. Thus, since $t$ occurs at least twice in $s$, we know that none of the $\$i$ are contained in $t$. Thus, $t$ must be a string in $\{0,1\}^{*}$. The only

\end{document}