\documentclass[11pt]{article}

\newcommand{\team}{Qian Long \\ Joseph W Presbrey \\ Mikhail Rudoy }
\newcommand{\ps}{ Problem Set 1 }

%\pagestyle{headings}
\usepackage[dvips]{graphics,color}
\usepackage{amsfonts}
\usepackage{amsthm}
\usepackage{amssymb}
\usepackage{amsmath}
\usepackage{latexsym}
\setlength{\parskip}{1pc}
\setlength{\parindent}{0pt}
\setlength{\topmargin}{-3pc}
\setlength{\textheight}{9.5in}
\setlength{\oddsidemargin}{0pc}
\setlength{\evensidemargin}{0pc}
\setlength{\textwidth}{6.5in}

\newtheorem*{Lemma}{Lemma}

\newcommand{\answer}[1]{
\newpage
\noindent
\framebox{
	\vbox{
		6.857 Homework \hfill {\bf \ps} \hfill \# #1  \\ 
		\team \hfill \today
	}
}
\bigskip

}


\begin{document}
\answer{1-3. Detecting Pad Reuse}
a) Define $R(c_i,c_j)$ to be the length of the longest run of identical bits in identical positions in ciphertexts $c_i$ and $c_j$.

First we will prove a lemma: 
\begin{Lemma}
Given that two ciphertexts $a$ and $b$ of length $n$ are properly encrypted, for any asymptotically positive polynomial $q(n)$, we have $\mathbb{P}\bigg( R(a,b) > \log_2(n)+\log_2\ln(n)\bigg) < \frac{1}{q(n)}$ for all sufficiently large $n$.
\end{Lemma}\begin{proof}
Since $a$ and $b$ are properly encripted, they are chosen uniformly at random from the space of bitstrings of length $n$. Then the xor of the two ciphertexts, $a \oplus b$, is also chosen uniformly at random from the space of bitstrings of length $n$. A bit of $a \oplus b$ is $0$ if and only if the corresponding bits of $a$ and $b$ are the same. The value $R(a,b)$ is the length of the longest sequence of bits in $a$ such that the corresponding sequence of bits in $b$ is the same. In other words, $R(a,b)$ is the length of the longest run of $0$s in $a \oplus b$. 

Since $a \oplus b$ is chosen uniformly at random, we can pretend that it was generated by a sequence of $n$ coinflips. Then if we identify $0$s in $a \oplus b$ with coins landing heads, we see that the useful fact given in the problem applies. The given fact asserts (in this case) that for any asymptotically positive polynomial $q(n)$ and for all $n \ge N$ for some $N$,  
$$\mathbb{P}\bigg(\log_2(n) - \log_2\ln\ln(n) \le R(a,b) \le log_2(n) + \log_2\ln(n)\bigg) \ge 1-\frac{1}{q(n)}$$
Then 
$$\mathbb{P}\bigg(\log_2(n) - \log_2\ln\ln(n) > R(a,b) \text{ or } R(a,b) > log_2(n) + \log_2\ln(n)\bigg) < \frac{1}{q(n)}$$
and finally 
$$\mathbb{P}\bigg(R(a,b) > log_2(n) + \log_2\ln(n)\bigg) < \frac{1}{q(n)}$$
for all $n \ge N$, as desired.\end{proof}

Let $l(n)$ be the number of ciphertexts we are given. Let $r(n)$ be any asymptotically positive polynomial. 

What we are interested is the value
$$\mathbb{P}\left(\max\limits_{1\le i < j \le l(n)}R(c_i,c_j) \le \log_2(n)+\log_2\ln(n)\right)$$
and in particular, we wish to show that it is greater than $1-\frac{1}{r(n)}$ for $n\ge n_0$ for some $n_0$.

Alternatively, since $$\mathbb{P}\left(\max\limits_{1\le i < j \le l(n)}R(c_i,c_j) > \log_2(n)+\log_2\ln(n)\right) = 1-\mathbb{P}\left(\max\limits_{1\le i < j \le l(n)}R(c_i,c_j) \le \log_2(n)+\log_2\ln(n)\right)$$
we must simply show that $\mathbb{P}\left(\max\limits_{1\le i < j \le l(n)}R(c_i,c_j) > \log_2(n)+\log_2\ln(n)\right) < \frac{1}{r(n)}$ holds for $n\ge n_0$ for some $n_0$.

In addition, we know that $$\mathbb{P}\left(\max\limits_{1\le i < j \le l(n)}R(c_i,c_j) > \log_2(n)+\log_2\ln(n)\right) =  \mathbb{P}\left(\begin{matrix} R(c_1,c_2) > \log_2(n)+\log_2\ln(n) & \text{or} \\ R(c_1,c_3) > \log_2(n)+\log_2\ln(n) & \text{or} \\ R(c_2,c_3) > \log_2(n)+\log_2\ln(n) & \text{or} \\ ... & \text{or} \\ R(c_{l(n)-1},c_{l(n)}) > \log_2(n)+\log_2\ln(n)  \end{matrix}\right)$$ and that $$\mathbb{P}\left(\begin{matrix} R(c_1,c_2) > \log_2(n)+\log_2\ln(n) & \text{or} \\ R(c_1,c_3) > \log_2(n)+\log_2\ln(n) & \text{or} \\ R(c_2,c_3) > \log_2(n)+\log_2\ln(n) & \text{or} \\ ... & \text{or} \\ R(c_{l(n)-1},c_{l(n)}) > \log_2(n)+\log_2\ln(n)  \end{matrix}\right) \le \sum_{i=1}^{l(n)-1}\sum_{j=i+1}^{l(n)}\mathbb{P}\bigg( R(c_i,c_j) > \log_2(n)+\log_2\ln(n)\bigg)$$

By the lemma, $\mathbb{P}\bigg( R(c_i,c_j) > \log_2(n)+\log_2\ln(n)\bigg) < \frac{1}{q(n)}$ for every $i$, $j$, and asymptotically positive polynomial $q(n)$, and for every $n\ge n_0$ for some $n_0$ dependent only on $q$. Let $q(n) = \frac{1}{2}r(n)l(n)(l(n)-1)$, and let $N$ be the associated value of $n_0$. Then for $n \ge N$ we see that $$\sum_{i=1}^{l(n)-1}\sum_{j=i+1}^{l(n)}\mathbb{P}\bigg( R(c_i,c_j) > \log_2(n)+\log_2\ln(n)\bigg) < \frac{l(n)(l(n) - 1)}{2}\times\frac{1}{q(n)}=\frac{l(n)(l(n) - 1)}{2q(n)}$$
where $$\frac{l(n)(l(n) - 1)}{2q(n)}=\frac{l(n)(l(n) - 1)}{2\frac{1}{2}r(n)l(n)(l(n)-1)} = \frac{1}{r(n)}$$

We can conclude that $\mathbb{P}\left(\max\limits_{1\le i < j \le l(n)}R(c_i,c_j) > \log_2(n)+\log_2\ln(n)\right) < \frac{1}{r(n)}$ for $n\ge N$, so $$\mathbb{P}\left(\max\limits_{1\le i < j \le l(n)}R(c_i,c_j) \le \log_2(n)+\log_2\ln(n)\right) > 1- \frac{1}{r(n)}$$ for $n\ge N$. Since $r(n)$ can be any asymptotically positive polynomial, we see that the length of the longest repeated bitstring $\bigg(\max\limits_{1\le i < j \le l(n)}R(c_i,c_j)\bigg)$ is, with high probability, at most $\log_2(n)+\log_2\ln(n)$, as desired.

\newpage

b) 

\newpage

c) 
\end{document}
